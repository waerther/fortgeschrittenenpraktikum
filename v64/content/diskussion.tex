\section{Diskussion}
\label{sec:Diskussion}
Als Kontrastwert ergab sich $K = 0.8$ unter einem Winkel von $47.69°$.
Die Abweichung von den theoretischen $45°$ Grad berechnet sich zu $5.98 \% $.
Diese Abweichung lässt sich damit erklären, dass die Spiegel nicht perfekt justiert worden sind.
Die Justierung gestaltete sich problematisch, da teilweise durch Rütteln bei dem gegenüberliegenden Versuch die Spiegel kleinste, aber deutlich messbare, Verrückungen erlitten.

Weiterhin wurde der Brechungsindex von Glas bestimmt.
Dieser wurde schließlich zu $1.53 \pm 0.031$ gemittelt.
Als Literaturwert wird $n_\text{lit} = 1.456$ angenommen.
Daraus ergibt sich dann für den gemessenen Wert eine Abweichung von $5.08 \%$.
Auch hier ist der größte Störfaktor die nicht optimal justierten Spiegel, wie am Kontrastwert zu erkennen ist.

Auch wurde der Brechungsindex von Luft gemessen.
Dieser wurde zu $n_{\text{norm}} = 1.005285 \pm 0.000023$ ermittelt.
Der Vergleich mit dem Theoriewert $n_\text{lit} = 1.0003$ zeigt eine Abweichung von $0.47\%$.