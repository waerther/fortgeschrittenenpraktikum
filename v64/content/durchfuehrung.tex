\section{Durchführung}
\label{sec:Durchführung}

Als Lichtquelle wird ein Helium-Neon-Laser mit einer Wellenlänge von $\lambda = 628.8 \unit{\nano\meter}$
Zunächst werden die Spiegel in dem Aufbau von \autoref{fig:aufbau1} so ausgerichtet, dass die Strahlen, die durch den PBSC getrennt worden sind, durch den PBSC laufen und wieder zusammengeführt werden.
Dafür lassen sich die Spiegel an verschiedenen Stellen jeweils in horizontaler und vertikaler Richtung justieren.
Der genaue Prozess der Justierung ist in der Versuchsanleitung \cite{v64} beschrieben und wird hier nicht wiederholt.

Nach der Justierung wird nun eine Kontrastmessung durchgeführt.
Zwischen dem Spiegel M2 und dem PBSC in \autoref{fig:aufbau1} ist eine Polarisationsscheibe eingestellt.
Für diese wird im Bereich $[30°,60°]$ in $2.5$er Schritten erhöht.
Hiernach wird bestimmt, unter welchem Winkel der maximale Kontrast vorzufinden ist.
Dafür wird zwischen den Stellen $8$ und $9$ in \autoref{fig:aufbau1} ein Glasplättchen eingesetzt und gedreht.
Die maximale und minimale Spannung, die durch ein angeschlossenes Multimeter erfasst werden, werden notiert.
Der Kontrast wird dann sofort berechnet und die Polarisationsscheibe wird so eingestellt, dass der Kontrast maximal ist.

Im Anschluss wird die Messung des Brechungsindex von Glas durchgeführt.
Dies geschieht durch die Differenzspannungsmethode.
Dabei werden die beiden Dioden an ein Zähler angeschlossen, der die Maxima zählt.
Die Maxima entstehen dann dadurch, dass an dem Glasplättchen in einem Bereich von $[0°,11°]$ gedreht wird.
Die Zählrate wird notiert und die Messung wird 10 mal wiederholt.

Anschließend wird eine Messung für den Brechungsindex von Luft durchgeführt.
Dafür wird eine Luftzelle zwischen Position $2$ und $3$ eingebaut und evakuiert.
Anschließend wird in etwa 50 $\unit{\milli\bar}$ Schritten Luft wieder in die Luftzelle zurückgelassen.
Die Maxima werden wieder gezählt und notiert.
Bei Atmosphärendruck angelangt wird die Messung wiederholt.
Insgesamt werden 3 Messungen durchgeführt. 