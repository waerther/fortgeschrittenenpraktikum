\section{Durchführung}
\label{sec:Durchführung}

Zunächst wird der Laser justiert.
Dafür wird der Justagelaser eingeschaltet und die Spiegel so justiert, dass der Strahl fokussiert auf den Fadenkreuzen liegt.
Nachdem dies gelungen ist, wird der HeNe - Laser eingeschaltet.
Nun sollte ein roter Strahl erkennbar sein.
Danach werden verschiedene Messungen durchgeführt.

\subsection*{Vermessung der Stabilität}
Zur Messung der Stabilität wird der Justierspiegel immer weiter auf der beweglichen Achse verschoben und dabei die Stromstärke auf dem Amperemeter abgelesen.
Dies wird mehrfach wiederholt, bis der Laser nicht mehr eingestellt werden kann und die maximale Länge überschritten ist.

\subsection*{Vermessung der $TEM_{xy}$ Moden}
Dann werden die verschieden Moden des Lasers untersucht.
Dafür wird zunächst die Intensität des Hauptmaximums in Abhängigkeit von der x-Achse genommen.
Es wird eine Photodiode angeschlossen und es wird eine Zerstreuungslinse eingebaut.
Die Stromstärke wird notiert, während die x-Achsen Position variiert wird.
Im Anschluss wird auch die $TEM_{01}$ Mode untersucht.
Hierfür wird ein Draht zwischen Spiegel und Diode eingeklemmt.
Mithilfe der Linse wird der Aufbau so eingestellt, dass sich eine Abbildung vergleichbar zu dem Plot von $TEM_{10}$ aus \autoref{fig:moden1} einstellt.
Danach wird dann wieder wie bei dem Hauptmaximum vorgegangen und die Stromstärke als Maß der Intensität notiert.

\subsection*{Nachweis der Polarisation}
Zur Überprüfung der Polarisation wird ein Polarisationsfilter zwischen Spiegel und Diode eingesetzt.
Dieser wird dann in $15°$ Schritten von null an bis $360°$ erhöht und die gemessene Stromstärke wird notiert.

\subsubsection*{Multimodenbetrieb und Frequenzanalyse}
Nun wird für verschiedene Längen eine Messung des Fourierspektrums durchgeführt.
Dafür wird eine sogenannte schnelle Photodiode angeschlossen, die mit einer Bandbreite von $1 \unit{\giga\hertz}$ messen kann.
Diese wird dann an einen Spektrumanalysator angeschlossen und die Peaks bei verschiedenen Frequenzen werden notiert.
Insgesamt wird diese Messung für 3 verschiedene Frequenzen wiederholt.

\subsection*{Messung der Wellenlänge}
Für die Messung der Wellenlänge wird ein Gitter zwischen Spiegel und Diode eingesetzt.
Das daraus resultierende Beugungsbild kann auf einem Schirm mit Abstand $d$ ausgemessen werden.
Hier werden dann die Abstände der Maxima zum Nullten Maximum mithilfe eines Maßbandes gemessen.
Auch der Abstand $L$ zwischen Schirm und Gitter wird gemessen.
Diese Messung wird für ein Gitter mit $100 \frac{\text{Linien}}{\unit{\milli\meter}}$ und $80 \frac{\text{Linien}}{\unit{\milli\meter}}$ durchgeführt.