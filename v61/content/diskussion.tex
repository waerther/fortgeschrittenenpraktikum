\section{Diskussion}
\label{sec:Diskussion}

Die bei der Überprüfung der Stabilitätsbedingung resultierenden Messdaten entsprechen in etwa der theoretischen Erwartung
einer quadrationen Kurve, wobei hierbei mehr Daten für eine bessere Überprüfung notwendig gewesen wären.
Dabei ergab sich die Schwierigkeit, 
dass das Nachjustieren der Resonatorspiegel und des Laserrohrs äußerst empfindlich ist
und das Aufrecherhalten der Stabilität nach der dritten Resonatorlänge nicht mehr möglich gewesen ist.
Die zu experimentellen Beobeachtungen der $\text{TEM}_{00}$- und $\text{TEM}_{01}$-Moden treffen ausgesprochen genau die
theoretische Voraussage. 
Die Messdaten der winkelabhägnigen Polarisationswinkel ergeben auch hier eine $\pi$-Periodizität der Intensität
mit den ungefähren Maxima bei $\phi = \qty{45}{\degree}$ und $\phi = \qty{225}{\degree}$.

Die Bestimmung der Wellenlängen für die zwei Gitter ergibt sich als relativ präzise mit den Abweichungen von
\begin{align*}
    \Delta \bar{\lambda_{80}} = \qty{3.52}{\percent} \quad \text{und} \\
    \Delta \bar{\lambda_{100}} = \qty{5.34}{\percent} \, .
\end{align*}