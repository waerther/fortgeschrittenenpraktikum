\section{Diskussion}
\label{sec:Diskussion}

Die bei der Überprüfung der Stabilitätsbedingung resultierenden Messdaten entsprechen in etwa der theoretischen Erwartung
einer quadrationen Kurve, bei dem ein Mininum bei $ x = \frac{R_1 + R_2}{2} = \frac{\qty{140}{cm} + \qty{140}{cm}}{2} = \qty{140}{cm}$ vorliegt.
Wobei hierbei mehr Daten für eine bessere Überprüfung notwendig gewesen wären.
Dabei ergab sich die Schwierigkeit, 
dass das Nachjustieren der Resonatorspiegel und des Laserrohrs äußerst empfindlich
und das Aufrecherhalten der Stabilität nach der dritten Resonatorlänge nicht mehr möglich gewesen ist.
Die dabei näherungsweise maximal erreichte Distanz konnte daher nur auf $\qty{95}{cm}$ bestimmt werden, 
obwohl der Laser theoretisch auf einer maximalen Distanz von $R_1 + R_2 =  \qty{140}{cm} + \qty{140}{cm} = \qty{280}{cm}$ stabil zu erwarten wäre. 
Hierbei ist zunächst auf einen instabilen Aufbau zu deuten. 

Die experimentellen Beobeachtungen der $\text{TEM}_{00}$- und $\text{TEM}_{01}$-Moden treffen ausgesprochen genau die
theoretische Voraussage von einer beziehungsweise zwei Gaußverteilungen.

Die Messdaten der winkelabhägnigen Polarisationswinkel ergeben auch hier eine $\pi$-Periodizität der Intensität
mit den ungefähren Maxima bei $\phi = \qty{90}{\degree}$ und $\phi = \qty{270}{\degree}$.

Der Multimoden-Betrieb

Die Bestimmung der Wellenlängen für die zwei Gitter ergibt sich als relativ präzise mit den Abweichungen von
\begin{align*}
    \Delta \bar{\lambda_{80}} = \qty{3.52}{\percent} \quad \text{und} \\
    \Delta \bar{\lambda_{100}} = \qty{5.34}{\percent} \, .
\end{align*}