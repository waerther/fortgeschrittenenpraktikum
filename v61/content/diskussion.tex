\section{Diskussion}
\label{sec:Diskussion}

Die bei der Überprüfung der Stabilitätsbedingung resultierenden Messdaten entsprechen nur, wenn überhaupt,
näherungsweise der theoretischen Erwartung
einer quadrationen Kurve, bei dem ein Mininum bei 
$ x = \frac{R_1 + R_2}{2} = \frac{\qty{140}{cm} + \qty{140}{cm}}{2} = \qty{140}{cm}$ vorliegen soll.
Wobei hierbei mehr Daten für eine bessere Überprüfung notwendig gewesen wären.
Dabei ergab sich die Schwierigkeit, 
dass das Nachjustieren der Resonatorspiegel und des Laserrohrs äußerst empfindlich
und das Aufrecherhalten der Stabilität nach der dritten Resonatorlänge nicht mehr möglich gewesen ist.
Die dabei näherungsweise maximal erreichte Distanz konnte daher nur auf $\qty{95}{cm}$ bestimmt werden, 
obwohl der Laser theoretisch auf einer maximalen Distanz von $R_1 + R_2 =  \qty{140}{cm} + \qty{140}{cm} = \qty{280}{cm}$ stabil zu erwarten wäre. 
Hierbei ist zunächst auf einen instabilen Aufbau zu deuten. 

Die experimentellen Beobeachtungen der $\text{TEM}_{00}$- und $\text{TEM}_{01}$-Moden treffen ausgesprochen genau die
theoretische Voraussage von einer beziehungsweise zwei Gaußverteilungen. 
Auch hier ist wichtig zu beachten, dass bei der Messung die Streulinse zur Erfassung der Moden verwendet wird,
da sonst der Strahldurchmesser nicht breit genug für die Photodiode ist.

Die Messdaten der winkelabhägnigen Polarisationswinkel ergeben auch hier eine $\pi$-Periodizität der Intensität
mit den ungefähren Maxima bei $\phi = \qty{90}{\degree}$ und $\phi = \qty{270}{\degree}$.
Das Brewster-Endfenster ermöglicht es, dass der reflektierte Strahl keine Komponente der Polarisation parallel zur reflektierten Oberfläche hat.
Dies führt schließlich zur senkrechten Polarisation und minimiert zusätzlich unerwünschte Reflexion von Licht der Resonatorspiegel.

Aus der Fourierspektroskopie geht hervor, 
dass sowohl die mittleren Frequenzabstände
als auch die abgeschätzen Verbreitungen innerhalb des Nennwerts des Neon-Übergangs von $\qty{1}{GHz}$ befinden.
Damit ist gezeigt, dass der Laser sich im Multimodenbetrieb befindet.
Und wie bereits in der Gleichung für die Schwebungsfrequenz ersichtlich, 
nehmen die Daten der Frequenzabstände mit höheren Resonatorlänge asymptotisch ab.

Die Bestimmung der Wellenlängen für die zwei Gitter ergibt sich als relativ präzise mit den Abweichungen von
\begin{equation*}
    \Delta \bar{\lambda}_{80} = \qty{3.52}{\percent} \quad \text{und} \quad 
    \Delta \bar{\lambda}_{100} = \qty{5.34}{\percent} \, .
\end{equation*}