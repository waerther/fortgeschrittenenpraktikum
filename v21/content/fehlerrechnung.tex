\section{Fehlerrechnung}
\label{sec:Fehlerrechnung}

Im Folgenden wird die allgemeine Fehlerrechnung und alle wichtigen Größen der entsprechenden Rechnung erklärt.
Die wichtigsten Werte dabei sind der 
\begin{align}
    \text{Mittelwert} \quad & \bar{x}  = \frac{1}{N} \sum_{i=0}^{n} x_i \quad \text{und die} \label{eq:mittelwert} \\
    \text{Standardabweichung} \quad & \sigma  = \sqrt{\frac{1}{N - 1 } \sum_{i=0}^{N} (x_i -  \bar{x})^2} \, . \label{eq:standartabweichung}
\end{align}

Dabei entspricht $N$ der Anzahl an Werten und $x_i$ ist jeweils ein mit einem Fehler gemessener Wert.
Es ergibt sich ebenfalls die statistische Messunsicherheit
\begin{equation}
    \increment \bar{x} = \frac{\sigma}{\sqrt{N}} = 
    \sqrt{\frac{1}{N(N - 1)} \sum_{i=0}^{N} (x_i -  \bar{x})^2} \, . \label{eq:messunsicherheit}
\end{equation} 

Entstehen mehrere Unbekannte in einer Messung, folgen daraus auch mehrere Messunischerheiten,
die in dem weiteren Verlauf der Rechnung berücksichtigt werden müssen.
Es gilt die \textit{Gaußsche Fehlerfortplanzung}
\begin{equation}
    \increment f(y_1 ,y_2 ,...,y_N ) = \sqrt{\left(\frac{\dif{f}}{\dif{y_{1}}} \increment y_{1}\right)^2
    + \left(\frac{\dif{f}}{\dif{y_{2}}} \increment y_{2}\right)^2 + ... + 
    \left(\frac{\dif{f}}{\dif{y_{N}}} \increment y_{N}\right)^2
    } \, . \label{eq:fehlerfortplanzung}
\end{equation}