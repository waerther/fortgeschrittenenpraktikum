\section{Diskussion}
\label{sec:Diskussion}

Die ermittelte horizontale Komponente des Erdmagnetfeldes von $B_\text{H} = \qty{23.9(5)}{\micro\tesla}$
weicht relativ gering von dem erwarteten Wert von etwa $\qty{20}{\micro\tesla}$ ab.
Das Vorjustieren der Messapparatur durch beispielsweise richtige Ausrichtung des Tisches oder
etwa das schmaler machen des ersten Peaks können dabei diese Abweichung bedingen.

Die bestimmten g-Faktoren liegen relativ nahe an den theoretischen Werten von
$g_{\text{F}_{85}}= 1/2$ und $g_{\text{F}_{87}}= 1/3$. 
Selbiges gilt für die berechneten Kernspins im Vergleich mit den theoretischen von
$I_{85}=3/2$ und $I_{87}=5/3$.
Auch hier bedingt größtenteils das Ablesen der Peaks die Abweichung.

Das abgeschätzte Isotopenverhältnis von $0.51$ entpsricht auch hier in etwa dem tatsächlichen Verhältnis.
