\section{Durchführung}
\label{sec:Durchführung}

Zur Durchführung des Versuch die Messapparatur in \autoref{fig:apparatur} zunächst vorjustiert.
\begin{figure}
    \centering
    \includegraphics[width=0.7\textwidth]{pictures/Zähler.jpg}
    \caption{Die Messapparatur nach dem Schaltbild in \autoref{fig:schaltung} am Ende der Messung.}
    \label{fig:apparatur}
\end{figure}

Zu Beginn werden die PMTs mithilfe der Oszilloskope auf ihre Funktion überprüft.
Die Ausgangsspannungen sollten an den beiden Ausgängen unterschiedliche Höhen aufweisen.
Die aus den Diskriminatoren ausgegebenen Schwellspannungen werden so eingstellt,
dass das Zählwerk etwa 30 Imulse pro Sekunde zählt.
Zur Justierung der Koinzidenzschaltung wird an den Diskriminatoren eine Pulsdauer von
$\Delta t_1 = \qty{20}{ns}$ und $\Delta t_2 = \qty{30}{ns}$ eingestellt.
Aufgrund von starken Signalschwankungen wird hierbei von den vorgegebenen Pulsdauern \cite{v01} abgewichen.
Nun werden die Verzögerungsleitungen in Schritten von \qty{1}{ns} variert 
und mithilfe des Impulszählers die Ereignisse für beide Pulsdauern nach \qty{20}{s} gezählt.
Dabei wird zunächst die erste Verzögerung maximiert und als größter Wert der negativen Skala gesetzt,
während die zweite auf \qty{0}{ns} eingestellt bleibt.
Nachdem die erste auf Null reduziert und festgesetzt wurde, wird die zweite Leitung schrittweise erhöht 
und die positive Skala der Verzögerung markiert.
Nach Verkabelung der restlichen Schaltung wird der Univibrator mit dem Oszilloskop auf die Suchzeit $T_\text{Such} = \qty{10}{\micro\second}$ eingestellt.
Der MCA wird anschließend mithilfe eines Doppelimpulsgenerators am Eingang der Koinzidenz justiert.
Dieser generiert Doppelimpulse mit einem variablen Zeitabstand bei einer Frequenz von \qty{1}{kHz}
und ermöglicht zu sehen, welche Kanäle für verschiedene Zeitabstände befüllt werden.
Hierbei wird für Sieben Abstände innheralb $\Delta t = \in [0,7 \, ; 5,4] \unit{\micro\second}$ gemessen.


Schließlich wird der Doppelimpulsgenerators wieder entfernt 
und die eigentliche Messung der Lebensdauer über einen PC mit einer Messsoftware gestartet.




