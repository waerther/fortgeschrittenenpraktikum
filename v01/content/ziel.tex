\section{Ziel}
\label{sec:Ziel}

In diesem Versuch wird die Lebensdauer kosmischer Myonen mithilfe der Zeitkorrelationsanalyse bestimmt.
Diese Methode dient zur Untersuchung der Verteilung der Zeitspannen zwischen dem
Zerfall aufeinanderfolgender Myonen.
Dabei wird anhand der Information über die Ankunftszeiten kosmischer Myonen
eine Wahrscheinlichkeitsverteilung der Zeitdifferenzen erstellt.
Hierbei wird ein Szintillator in Verbindung mit zwei Photomultipliern als Detektor verwendet,
um mithilfe einer Logikschaltung anhand von erzeugten Signalen die Myonenereignisse zu identifizieren.
Die Zeitdifferenz zwischen dem Start- und dem Stoppsignal wird gemessen und in einem Vielkanalanalysator aufgetragen, 
um das Spektrum der Zeitdifferenzen zu erhalten. 
Aus diesem Spektrum kann die Lebensdauer der Myonen bestimmt werden.