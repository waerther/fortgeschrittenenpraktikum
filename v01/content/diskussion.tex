\section{Diskussion}
\label{sec:Diskussion}

In der Kalibrierung der Messgeräte ergaben sich einige Schwierigkeiten.
Zum einen ließ sich mithilfe des Schraubendrehers keine Pulsdauer von $10 \unit{\nano\second}$ einstellen, weshalb auf eine Messung bei $30 \unit{\nano\second}$ zurückgegriffen wurde.
Allerdings ist auch die $20 \unit{\nano\second}$ Messung nicht gut verlaufen, da hier große Schwankungen erkennbar sind, obwohl sich hier theoretisch eine Art Maximum ausbilden sollte.
Dies konnte nicht beobachtet werden.
Zudem wurde die Messung nicht mit einer genügend großen Verzögerung gestartet.
Dies wollten die Experimentatoren dann nachholen, jedoch kam es dann zu Schwierigkeiten bei der Apparatur, die die Messung verhinderte.
Mit Hilfe des Betreuers des Versuchs gelang es die Messungen wieder aufzunehmen.
Allerdings konnte die Messung bei $20 \unit{\nano\second}$ nicht beendet werden.
Aus diesem Grunde ist die Halbwertsbreite nur grob abschätzbar, da es sich nur erahnen lässt, wie breit diese tatsächlich ist.
Bei der Messung zu $30 \unit{\nano\second}$ kam es auch zu massiven Schwankungen, jedoch trat ein weiteres Problem auf.
Die Messdaten flachten kaum ab, so dass nach einer angemessenen Anzahl an Messwerten die Experimentatoren die Messung als abgeschlossen ansahen.
Auch dies ist in \autoref{fig:30ns_plot} zu erkennen.
Dadurch ist auch die Abschätzung der Halbwertsbreite bei $30 \unit{\nano\second}$ mit Vorsicht anzusehen.
Schlussendlich wurde dann eine Pulsdauer von $20 \unit{\nano\second}$ und eine Verzögerung von $\Delta t = -3$ eingestellt, da die Experimentatoren hier bei der INterpretation der Messwerte einen Fehler machten.

Die Lebenszeit wurde zu $\tau = tbd$ bestimmt.
Als Literaturwert wird $\tau_\text{Lit} = 2,2 \unit{\micro\second}$ angegeben.
Damit ergibt sich eine Abweichung von $tbd$.
tbd.
Der Untergrund durch den Fit ergab $U = tbd$ und aus der theoretischen Überlegung $U_\text{theo} = tbd$.
Die prozentuale Abweichung beträgt $\Delta U = tbd$.