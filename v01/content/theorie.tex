\section{Theorie}
\label{sec:Theorie}

Myonen sind instabile Elementarteilchen der Gruppe der Leptonen. 
Sie tragen ein negative Ladung, sind etwa 200-mal schwerer als Elektronen 
und haben eine Lebensdauer von ca. \qty{2,2}{\micro\second}.
Sie entstehen durch die Wechselwirkung hochenergetischer kosmischer Strahlung
mit den oberen Schichten der Atmosphäre,
auch bekannt als sekundäre kosmische Strahlung.
Aus Protonenschauern gehen Pionen hervor,
die über die Zerfälle
\begin{equation*}
    \pi^{+} \rightarrow \mu^{+}+\nu_\mu \qquad \text {und} \qquad \pi^{-} \rightarrow \mu^{-}+\bar{\nu}_\mu
\end{equation*}
schließlich Myonen erzeugen.

