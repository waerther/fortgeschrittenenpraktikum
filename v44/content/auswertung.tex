\section{Auswertung}
\label{sec:Auswertung}

Im nun Folgenden werden die aufgenommenen Messdaten der verschiedenen Scans ausgewertet. 


\subsection{Detektorscan}
\label{sec:Detektorscan}

Zu Beginn wird anhand des Detektorscans die Halbwertsbreite und die maximale Intensität bestimmt.
Hierfür wird eine Gaußfunktion der Form
\begin{equation}
    I(\theta)=\frac{a}{\sqrt{2 \pi \sigma^2}} \cdot \exp \left(-\frac{(\theta-\mu)^2}{2 \sigma^2}\right)+b
\end{equation}
als Regression an die Messdaten verwendet,
die in \autoref{fig:Detektorscan} zu sehen ist.


Die daraus resultierenden Parameter ergeben sich zu 
\begin{align*}
    a & = \qty{4.34(6)e4}{} \, , \\
    b & = \qty{9.5(1.9)e3}{} \, , \\
    \sigma & = \qty{-0.0361(5)}{} \, \text{und} \\
    \mu & = \qty{0.0011(5)}{\degree} \, .
\end{align*}

Mittels der Phyton-Bibliothek \textit{SciPy} \cite{scipy} ergeben sich die Halbwertsbreite FWHM
und die maximale Intensität $I_\text{max}$ Zu
\begin{align*}
    \text{FWHM} & = \qty{0.0849}{\degree} \, \text{und} \\
    I_\text{max} & = \num{4.89e5} \, \# \, / \, \unit{\second} \, .
\end{align*}


\subsection{Z-Scan}

