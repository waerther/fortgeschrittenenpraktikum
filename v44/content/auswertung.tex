\section{Auswertung}
\label{sec:Auswertung}

Im nun Folgenden werden die aufgenommenen Messdaten der verschiedenen Scans ausgewertet. 


\subsection{Detektorscan}
\label{sec:Detektorscan}

Zu Beginn wird anhand des Detektorscans die Halbwertsbreite und die maximale Intensität bestimmt.
Hierfür wird eine Gaußfunktion der Form
\begin{equation}
    I(\theta)= A \cdot \exp \left(-\frac{(\theta-\mu)^2}{2 \sigma^2}\right)+b
\end{equation}
an die Messdaten gefittet.
Die daraus resultierenden Parameter ergeben sich zu 
\begin{align*}
    a & = \qty{4.34(64)e4}{} \, , \\
    b & = \qty{9.48(1.87)e3}{} \, , \\
    \sigma & = \qty{3.614(0.054)e-2}{} \, \text{und} \\
    \mu & = \qty{1.07(0.51)e-3}{\degree} \, .
\end{align*}
Die Halbwertbreite und die maximale Intensität ergeben sich zu
\begin{align*}
    I_\text{max} &= \qty{4.89e5}{} \\
    &\text{und} \\
    \text{FWHM} &= \qty{8.63e-2}{\degree} \, .
\end{align*}
Die Regression ist in \autoref{fig:Detektorscan} zu sehen.
\begin{figure}
    \centering
    \includegraphics[width = 0.65 \linewidth]{build/Detektorscan.pdf}
    \caption{Fit der Gaußfunktion an die Messdaten.
    Die gestrichelten Linien deuten die Halbwertsbreite an.}
    \label{fig:Detektorscan}
\end{figure}

\subsection{Z-Scan}

\begin{figure}
    \centering
    \includegraphics[width = 0.65 \linewidth]{build/z_scan.pdf}
    \caption{Messdaten des Z-Scans und Analyse zur Strahlbreite.
    DIe gestrichelten Linien geben an, in welchem Bereich sich die Probe durch den Strahl bewegt und
    lassen somit eine Deutung der Strahlbreite zu. }
    \label{fig:z_scan}
\end{figure}

Mithilfe des Z-Scans kann die Strahlbreite des Röntgenstrahls abgeschätzt werden.
Die Messdaten zum Versuch und die Analyse sind in \autoref{fig:z_scan} zu finden.
Die Strahlbreite ergibt sich zur $d_0 = 0{,}14 \unit{\milli\meter}$.

\subsection{X-Scan}

\begin{figure}
    \centering
    \includegraphics[width = 0.65 \linewidth]{build/x_scan.pdf}
    \caption{Zu sehen sind die Messdaten des X-Scans und der stetige Abfall der Intensität.}
    \label{fig:x_scan}
\end{figure}

Desweiteren wird der X-Scan zur Bestimmung der Position der Probe durchgeführt.
Die Messdaten finden sich in \autoref{fig:x_scan} wieder.
Im Anschluss der Messung wurde eine beliebige Position innerhalb des Intervalls der Probe gewählt.

\subsection{Rockingscan}

\begin{figure}
    \centering
    \includegraphics[width = 0.75 \linewidth]{build/rockingscan.pdf}
    \caption{Rockingscan zur Analyse des Auflagewinkels der Probe.
    Links ist die grobe Messung zu sehen.
    Rechts ist bereits justiert worden. Es ist zu sehen, dass das Dreieck nun gleichschenklig ist.
    Somit ist die Justage durch den Rockingscan gelungen.}
    \label{fig:rockingscan}
\end{figure}

Wie in \autoref{sec:Durchführung} beschrieben, wird jetzt ein Rockingscan durchgeführt, um die Verkippung der Probe zu korrigieren.
Diese Analyse wird auch genutzt um den Geometriewinkel abzuschätzen.
Diese Abschätzung und die Messdaten sind in \autoref{fig:rockingscan} zu sehen.
Der Geometriewinkel ergibt sich dabei zu $\theta_\text{Geom} \approx 0{,}35 °$.
Der theoretische Geometriewinkel ergibt sich durch 
\begin{equation*}
    \theta_\text{Geom, Theo} = arcsin\left(\frac{d_0}{D}\right) = 0{,}401 \, \unit\degree \, .
\end{equation*}
Dabei ist $D = 20 \, \unit{\milli\meter}$ und beschreibt die Ausmaße der Probe.

\subsection{Messung}

\begin{figure}
    \centering
    \includegraphics[width = 0.8 \linewidth]{build/messung1.pdf}
    \caption{Reflektivität aufgetragen gegen den Winkel.
    Weiterhin sind die Minima eingezeichnet, welche zur Berechnung der Abstände der Kiessig-Oszillationen benötigt werden.
    Die Formel musste weiterhin noch durch den Geometriewinkel korrigiert werden.
    Die Näherung des Freselkoeffizienten ist ebenfalls eingezeichnet.}
    \label{fig:messreihe1}
\end{figure}

Nachdem die Apperatur nach \autoref{sec:Durchführung} justiert wurde, wird nun die tatsächliche Messung durchgeführt.
Die Messdaten sind in \autoref{fig:messreihe1} zu sehen.
Hier ist die normale, \enquote{direkte} Messung abzüglich der diffusen Messung aufgetragen.
Es sei angemerkt, dass hier bereits die Intensität in die Reflektivität umgewandelt wurde.
Diese berechnet sich durch $R = \frac{I}{ 5 \cdot I_\text{max}}$.
Die Minima der Kiessig Oszillationen sind ebenfalls eingezeichet.
Berechnet wurden sie durch die Python Bibliothek \textit{SciPy}, beziehungsweise durch \textit{scipy.signal.argrelextrema}.
Dabei wurden nur Minima führender Ordnung angegeben, da in einem größeren Winkelbereich die Messung zu ungenau wird.
Nun müssen die Abstände dieser Minima berechnet werden und gemittelt werden.
Diese Rechnung ergibt
\begin{equation*}
    \Delta \alpha = (9{,}5 \pm 2{,}2) \cdot 10^{-4} \unit\radian \, .
\end{equation*}
Mit diesen Größen kann nun die Schichtdicke $d$ bestimmt werden.
Dafür wird die allgemein bekannte Wellenlänge der $K_\alpha$ Linie von $\lambda = 1{,}541 \cdot 10^{-10}$ in \autoref{eq:schichtdicke} eingesetzt.
Die Rechnung ergibt
\begin{equation*}
    d = (8{,}1 \pm 1{,}8) \cdot 10^{-8} \, \unit\meter \, .
\end{equation*}

\begin{figure}
    \centering
    \includegraphics[width = 0.8 \linewidth]{build/messung2.pdf}
    \caption{Es wird die gemessene Reflektivität verglichen mit dem Fit des Parratt Algorithmus.
    Außerdem sind die berechneten kritischen Winkel eingezeichnet.}
    \label{fig:messreihe2}
\end{figure}

Nun wird mithilfe des Parratt-Algorithmus eine Theoriekurve an die vorhandenen Messdaten gefittet.
Der entsprechende Fit ist in \autoref{fig:messreihe2} zu sehen.
Der Fit ergibt die Parameter
\begin{align*}
    d & = \qty{8.545(0.14)e-08}{\unit\meter} \, , \\
    \delta_2 & = \qty{1.52(0.51)e-06}{} \, , \\
    \delta_3 & = \qty{5.3(0.017)e-6}{} \, , \\
    \sigma_1 & = \qty{9.06(7.7)e-10}{\unit\meter} \, \text{und} \\
    \sigma_2 & = \qty{8.19(2.71)e-10}{\unit\meter} \, . \\
\end{align*}
Der verwendete Code ist im Anhang beigefügt.
Die Annahmen sind hierbei, dass das System aus einer Schicht Polysterol und einer Schicht Silizium besteht.
Beide haben dadurch verschiedene Brechungsindizes.
Desweiteren müssen einige heuristische Annahmen bei dem verwendeten Ansatz genutzt werden, um einen guten Fit zu erreichen.
Dies ist durch Startbedingungen und Grenzwerten bei der verwendeten Fitmethode geschehen.
Auch die kritischen Winkel konnten so ermittelt werden.
Diese ergaben sich nach $\alpha_{\text{c}, i} \approx \sqrt{2 \delta_i}$ zu
\begin{align*}
    \alpha_{\text{c, PS}} &= 0{,}109 \, ° \\
    &\text{und} \\
    \alpha_{\text{c, Si}} &= 0{,}203 \, ° \, .
\end{align*}