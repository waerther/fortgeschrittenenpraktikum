\section{Durchführung}
\label{sec:Durchführung}

In dem hier vorhandenen Aufbau liegt ein \enquote{D8-Labordiffraktometer} der Firma Bruker-AXS vor.
Dabei entsteht die Röntgenstrahlung gemäß \autoref{sec:Röntgenstrahlung} in einer Kupferanodenröhre, die mit einem Strom von $I=35 \, \unit{\milli\ampere}$ 
und einer Spannung von $U = 40  \, \unit{\kilo\volt}$ betrieben wird.
Der Strahl wird mithilfe eines Göbelspiegels gebündelt und monochromatisiert.
Die Wellenlänge des Röntgenstrahls beträgt dann $\lambda = 1{,}55  \, \unit{\angstrom}$, da es sich um die $K_\alpha$-Linie von Kupfer handelt.

\subsection{Justage} \label{sec:Justage}

Bei der Justage kommen insgesamt vier verschiedene Arten von Scans zum Einsatz.
Begonnen wird dabei mit dem \textbf{Detektorscan}.
Diese Messung dient dazu, die Nulllage des Detektors zu finden und damit die Intensität des Strahles zu maximieren.
Die Probe wird aus dem Strahlengang entfernt, damit der Strahl ungehindert durch den Raum propagieren kann.
Dann wird die Position gesucht, in der die Intensität des Strahles maximal ist und als neue Nulllage definiert.

Im Anschluss wird ein sogenannter \textbf{Z-Scan} durchgeführt.
Wie der Name bereits andeutet, handelt es sich um einen Scan zur Justage der z-Achse.
Dafür wird zunächst die Probe wieder in den Strahlengang geschoben.
Die z-Position der Probe wird so lange variiert, bis die Intensität mit Probe auf die halbe Intensität ohne Probe abgesunken ist, also $I_{z,\text{Probe}} = \frac{I_\text{max}}{2}$.
Auf dieser $z$ Position wird dann der Probentisch gehalten.

Nun wird ein \textbf{X-Scan} durchgeführt.
Dabei wird der Probentisch senkrecht zum Strahl (x-Achse) bewegt.
Hierbei wird überprüft, das der Strahl auch wirklich die Probe und nicht den Tisch trifft.
Dies wird dadurch getestet, dass die Intensität des Strahls abnimmt, während er durch die Probe abgeschirmt wird.
Im Anschluss wird eine beliebige Position in diesem Minimum gewählt.   

Nun wird zu dem \textbf{Rockingscan} übergegangen.
Dabei handelt es sich um einen Scan, bei dem die Röhre und Detektor in einer festen Achse um die Probe rotiert werden.
Durch diesen Scan kann eine Mögliche Verkippung der Probe detektiert werden.
Außerdem kann durch ihn die Probe (durch mögliche Fehlpositionierung auf dem Tisch) in den Drehpunkt des Diffraktometers gebracht werden.
Zwischen Detektor und Probe wird dabei ein Winkel von $2 \Theta = 0$ beibehalten.
Die Messung wird durch kleine Schritte im Bereich $[-1°;1°]$ durchgeführt.
ist das Maximum in der Intensität gefunden, wird der entsprechende Winkel notiert.

Nun kommt es zur Feinjustage der Probe.
Dabei wird zunächst ein Z-Scan und dann zwei weiterere Rockingscans (bei $2 \Theta = 0{,}3 \, °$ und $2 \Theta = 0{,}5 \, °$) durchgeführt.
Nun kann zur eigentlichen Messung übergegangen werden.

\subsection{Tatsächliche Messung}
Für die tatsächliche Messung wird der Winkel des Detektors und der Röhre auf null gestellt und $2 \Theta$ ebenfalls.
Dann wird unter dem Modus \enquote{Omega/2Theta} im Bereich $[0 \, ; \, 2{,}5]$ eine Messung mit einer Zeit pro Messwert von $5  \, \unit\second$ aufgenommen.
Nun kann das Messprogramm gestartet werden.
Im Anschluss daran wird noch eine \enquote{Diffuse}-Messung durchgeführt.
Dafür wird der Winkel des Detektors auf $0{,}1 \, °$ gestellt.
Der Rest wird analog zur Messung davor eingestellt und das Messprogramm kann gestartet werden.