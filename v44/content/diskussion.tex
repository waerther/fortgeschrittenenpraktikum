\section{Diskussion}
\label{sec:Diskussion}

Der Rockingscan ergab einen Geometriewinkel von $\theta_\text{Geom} = 0{,}35 \, \unit\degree$ und theoretisch
$\theta_\text{Geom, Theo} = 0{,}401 \, \unit\degree$.
Damit ergibt sich eine Abweichung zwischen den beiden Werten von $\Delta \theta = 14{,}47 \, \%$.
Diese Abweichung könnte dadurch erfolgen, dass die Apperatur noch nicht vollständig justiert war und die Seiten nicht vollständig symmetrisch waren.
Dies ist im entsprechenden Plot auch ersichtlich.

Die Schichtdickte wurde einmal durch die Abstände der Minima mithilfe der Kiessig Oszillationen und durch den Parratt-Algorithmus-Fit bestimmt.
Die Werte lauten
\begin{align*}
    d_\text{Kiessig} &=  (8{,}1 \pm 1{,}8) \cdot 10^{-8} \, \unit\meter\\
    & \text{und} \\
    d_\text{Parratt} & = \qty{8.495(0.0279)e-08}{} \, \unit\meter \, .
\end{align*}
Zwischen den Werten ergibt sich eine Abweichung von $4{,}88\, \%$.
Es ist anzumerken, dass die Ergebnisse des Parratt Algorithmus nach der hier verwendeten Implementation sehr stark mit den Anfangswerten und Grenzwerten der Variablen korreliert.
Dies liegt daran, dass die Werte von $R$ sich stark an null angleichen.
Dadurch werden die Algorithmen numerisch instabil.
Deshalb wurde viele heuristische Annahmen getroffen, damit die Ergebnisse den Erwartungen entsprechen.
Dies äußerte sich durch einsetzten vieler verschiedener Werte in die Anfangswerte und Grenzwerte.

Nun werden die ermittelten Werte der Dispersionen für Polysterol und Silizium mit den Literaturwerten \cite{v44} verglichen.
\begin{align*}
    \delta_2 &= \qty{1(0.01)e-08}{} & \delta_{\text{PS,lit}} &= 3{,}5 \cdot 10^{-6} & \Delta {\delta_2} &\approx 99{,}71\, \% \\
    \delta_3 &= \qty{4.67(0.017)e-8}{} & \delta_{\text{SI,lit}} &= 7{,}6 \cdot 10^{-6} & \Delta {\delta_3} &\approx 99{,}39\, \%
\end{align*}
Diese Abweichungen sind sehr drastisch. Gerade, da die Parameter stark an der null liegen.
Auch hier ist es vermutlich auf die händische Feinjustierung zurückzuführen.
Für die kritische Winkel ergibt sich eine Abweichung zu den Theoriewerten \cite{tolan_xray} von
\begin{align*}
    \alpha_{\text{c,PS}} &= 0{,}092° & \alpha_{\text{c,PS,lit}} &= 0{,}153° & a_{\alpha_{\text{c,PS}}} &\approx 39{,}87 \, \%  \\
    &&&\text{und}&& \\
    \alpha_{\text{c,Si}} &= 0{,}197° & \alpha_{\text{c,Si,lit}} &= 0{,}223° & a_{\alpha_{\text{c,Si}}} &\approx 13{,}2 \, \% \, .
\end{align*}